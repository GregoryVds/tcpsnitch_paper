\section{UDP sockets}\label{sec:udp}

We first analyze how UDP is used by the 31 applications in our dataset that
open at least one \texttt{SOCK\_DGRAM} socket. Note that those UDP sockets are
explicitly requested by the applications themselves since \tcpsnitch does not
track \texttt{getaddrinfo()} and related functions that are part of
\texttt{libc}.

\begin{table}[]
\centering
\begin{tabular}{ll}
85\% & Get info about network with \texttt{ioctl()} \\
8\% & \texttt{connect()} but does not exchange data \\
6\% & Send or receive data \\
1\% & Other usages
\end{tabular}
\caption{UDP sockets usage. \textmd{Most UDP sockets do not send or receive 
\emph{any} data but get information about the network environment using ioctl().}}
\label{tab:udp_sockets_usage}
\end{table}

Looking at the amount of data sent/received over those UDP sockets, we
noticed that 85\% of the opened \texttt{SOCK\_DGRAM}
sockets do not send or receive \emph{any} data. Those sockets are
created to retrieve information about the networking environment using
\texttt{ioctl()}. While a single application opened 30\% of these
sockets, 15 applications use UDP and never send UDP data.
Table~\ref{fig:udp_ioctls} details the main \texttt{ioctl()}
requests. Although those \texttt{ioctl()} apply to any socket, we
suspect that applications perform them on UDP sockets because they are
cheaper than their TCP counterpart.

\begin{table}[]
    \centering
    \begin{tabular}{lll}
        \textbf{} & \textbf{Request} & \textbf{Purpose (get dev *)}      \\
        44\%   & SIOCGIFADDR      & Address                   \\
        25\%   & SIOCGIFNAME      & Name                      \\
        20\%   & SIOCGIFFLAGS     & Active flag word          \\
        5\%    & SIOCGIFNETMASK   & Network mask              \\
        5\%    & SIOCGIFBRDADDR   & Broadcast address         \\
        1\%    & Others           & N/A
    \end{tabular}
    \caption{ioctl() requests breakdown. 85\% of the UDP sockets use these
    requests to get information about the network devices.}
    \label{fig:udp_ioctls}
\end{table}

Overall, 16 applications sent or received data over UDP: 5 are
video-telephony apps such as Google Hangout or Skype, 4 are video
or music streaming applications such as Spotify or Netflix and 3 are
Google applications likely using QUIC like Chrome or Google Plus. The rest
are various applications that only exchange a few hundred bytes such as
Shazam or Angry Birds. Applications mainly use \texttt{sendto()} and
\texttt{recvfrom()} to send or receive data. We observed that 29\% of the
receiving calls set the \texttt{MSG\_PEEK} to peek on the receiving queue without
removing data and that 0.6\% of the sending calls set the \texttt{MSG\_NOSIGNAL}
flag to prevent a \texttt{SIGPIPE} from being raised in case of error.
We did not find any indication of the usage of the other flags on
\texttt{SOCK\_DGRAM} sockets. We noticed that Messenger uses the
\texttt{SIOCGSTAMP} iotcl during video calls roughly every second
\texttt{recvfrom()}. This ioctl allows round trip time measurements. Among the
\texttt{SOCK\_DGRAM} sockets that we observed, only 6\% sent or received data.
It is interesting to note that 8\% of the \texttt{SOCK\_DGRAM} sockets issued
a \texttt{connect()} without sending or receiving any data.

Multicast is one of the use cases for UDP sockets. We observed 8
applications that used UDP sockets to send multicast packets but
only 2 applications joined multicast groups. These 2 applications use
multicast to discover other similar applications on the same LAN, e.g.
using the Simple Service Discovery Protocol. A typical example are
streaming applications that allow to discover another device where
audio/video can be streamed over the network.
