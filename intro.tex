\section{Introduction}

The socket API was introduced together with release 4.2 of
the BSD Unix distribution that included a functional TCP/IP stack
\cite{Quarterman:BSD}. This API allows applications to interact with
the underlying networking stack. When the socket API
was designed, TCP/IP was one family of network protocols among many
others and it was important to abstract those protocol families.
The heart of the socket API is  a set of basic system calls
including \texttt{socket}, \texttt{bind}, \texttt{connect},
\texttt{accept}, \texttt{close}, \texttt{listen}, \texttt{send},
\texttt{receive},\ldots Those system calls interact with the
underlying network implementation that is part of the operating
system kernel. The socket API was not the only approach to
interact with the network stack. The STREAM API, based on
\cite{ritchie1984unix} was extended to support the TCP/IP protocol
stack and used in Unix System V \cite{rago1993unix}.

Over the years, the popularity of the socket API grew in parallel
with the deployment of the global Internet. Nowadays, the
socket API is considered by many as the standard API for
simple networked applications. Several popular textbooks are entirely
devoted to this API \cite{Stevens:UNP,donahoo2001pocket}. Given the
importance of web-based applications, many developers do not interact
directly with the socket API anymore but rely on higher-level
abstractions. For example, programming languages such as Java or Python
include libraries exposing URL and implementations of
HTTP/HTTPS. For C developers, libraries such as \texttt{libcurl} 
also provide higher level abstractions.

During the last 30 years, the socket API has evolved, with new features added
over the years. Some were dedicated to the support of specific features, such
as ATM \cite{almesberger1998using} or Quality of Service
\cite{abbasi2002quality}. Other extensions \cite{Provos_Scalable:2000,
gammo2004comparing}
focused on improving the
interactions between applications and the underlying stack through
\texttt{poll/select}, \texttt{epoll},\ldots On Linux, new system
calls to directly send pages or entire files (like \texttt{sendfile})
were added. Furthermore, socket extensions have been defined
for each new transport protocol \cite{schier2012using,natarajan2009sctp}.
Socket extensions have also been proposed to deal with multihoming
\cite{Schmidt:intents} and specific APIs have been implemented on top
of Multipath TCP \cite{hesmans2015smapp,hesmans2016enhanced}.

Recently, the Internet Engineering Task Force created the 
Transport Services (taps) working group whose main objective is
\emph{to help application and network 
stack programmers by describing an (abstract) interface for applications 
to make use of Transport Services}. Although this work will focus on
abstract transport services, understanding how the current APIs
are used by existing applications will help in designing generic transport
services that correspond to their needs.

We propose \texttt{TCPSnitch}, an open-source software that collects
detailed traces of the interactions between networked applications and
the Linux TCP/IP stack and sends them to a publicly available database exposing
various statistics. This paper is organized as follows. We first
describe \tcpsnitch in section~\ref{sec:tcpsnitch}. Then we present in
section~\ref{sec:dataset} the traces that we collected from 90
different Android applications. Section~\ref{sec:udp} analyses in more
details the utilization of UDP sockets by those applications while
section~\ref{sec:tcp} focuses on the TCP sockets. We summarize the
main findings of this work and future work in section~\ref{sec:conclusion}.
